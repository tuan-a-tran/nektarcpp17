\input{tutorial-preamble.tex}

\title{NekMesh: High-order mesh generation (NACA)}

\begin{document}

%\frontmatter

% Render pretty title page if not building HTML
\ifdefined\HCode
\begin{center}
    \includegraphics[width=0.1\textwidth]{img/icon-blue.png}
\end{center}
\maketitle
\begin{center}
  \huge{Nektar++ Tutorial}
\end{center}
\else
\titlepage
\fi

\clearpage

\chapter{Introduction}

This tutorial covers a new feature of NekMesh introduced in the two-dimensional
mesh generator. This feature focuses on the creation of a NACA profile surrounded
by a rectangular bounding box as well as its meshing.

\begin{tipbox}
  This feature is an advanced functionality of NekMesh and the reader is advised
  to first complete the standard NekMesh tutorial \inlsh{NekMesh: High-order mesh
  generation (from STEP)} before proceeding. This chapter however includes a
  reminder of the steps required to generate a mesh and run the solver.
\end{tipbox}

To achieve the goals of this tutorial you will need an installation of the
latest \texttt{master} version of Nektar++, available at
\href{http://www.nektar.info}{nektar.info}. This is because the features you
will be using are relatively new and have yet to be included in the latest
release. Standard installation instructions are applicable as described in the
\href{http://www.nektar.info/downloads/8}{user guide}, with one addition that
CMake should be configured to include \inlsh{NEKTAR\_USE\_MESHGEN=ON}. This will
activate the installation of the relevant meshing routines.

\begin{tipbox}
  When using the mesh generation routines Nektar++ will require an OpenCascade
  installation to be present on your machine. Nektar++ is capable of downloading
  and installing OpenCascade for you, but this is highly not recommended as the
  installation time is very large. OpenCascade Community Edition (OCE) can be
  installed through package managers:

  \begin{itemize}
    \item \textbf{Homebrew:} \texttt{brew install homebrew/science/oce}
    \item \textbf{MacPorts:} \texttt{port install oce}
    \item \textbf{Ubuntu/Debian:} \texttt{sudo apt-get install liboce-foundation-dev liboce-modeling-dev}
    \item Other Linux installations have a version of OCE in their package manager,
          usually called \texttt{liboce}.
  \end{itemize}

  Nektar++ will be able to pick up on any of these OCE
  installations. Installation of NekMesh on Windows with mesh generation is not
  possible due to the OpenCascade requirements at this time.
\end{tipbox}

This tutorial additionally requires:
\begin{itemize}
    \item Nektar++ IncNavierStokesSolver with pre- and post-processing tools,
    \item A visualisation tool such as \href{http://www.paraview.org}{\underline{Paraview}}
    or
    \href{https://wci.llnl.gov/simulation/computer-codes/visit/downloads}{\underline{VisIt}}
\end{itemize}

\section{Goals}
After the completion of this tutorial, you will be familiar with:
\vspace{-0.5cm}
\begin{itemize}
\item the generation of a 2D mesh from the NACA generator;
\item the visualisation of the mesh in Paraview or VisIt
\item running a simple simulation on this mesh with IncNavierStokesSolver; and
\item the post-processing of the data and the visualisation of the results in
Paraview or VisIt.
\end{itemize}

\begin{tutorialtask}
Prepare for the tutorial. Make sure that you have:
\begin{itemize}
  \item Installed and tested \nektar v\nekver{} compiled from source. We will
  refer to the directory where you installed \nektar as \inlsh{\$NEK} for
  the remainder of the tutorial.
  \item Make a directory of your choosing, for example \inlsh{tutorial}, and
  download the tutorial files from
  \relurl{mesh-generation-2d-naca.tar.gz}{mesh-generation/2d-naca} into this
  directory.

  \item Unpack the tutorial files by using \tutorialcommand{tar -xzvf
    mesh-generation-2d-naca.tar.gz} to produce a directory \inlsh{2d-naca} with
  subdirectories called \inlsh{tutorial} and \inlsh{complete}.
\end{itemize}
\end{tutorialtask}


The tutorial folder contains:
\begin{itemize}
\item an .mcf file;
\item a session file which will run the NACA mesh.
\end{itemize}

\chapter{NACA wing}

An additional feature of the 2D capabilities of NekMesh is to generate meshes of
NACA wings without the need for a CAD file. By listing the \texttt{CADFile}
parameter as a 4 digit NACA code in the .mcf, NekMesh will generate the NACA
wing. Additional parameters about the domain are required, that define the
minimum and maximum x- and y-coordinates for the grid and the angle of attack of
the wing:

\begin{lstlisting}[style=XMLStyle]
    <P PARAM="Xmin"     VALUE="-1.0"    />
    <P PARAM="Ymin"     VALUE="-1.0"    />
    <P PARAM="Xmax"     VALUE="3.0"     />
    <P PARAM="Ymax"     VALUE="1.0"     />
    <P PARAM="AOA"      VALUE="15.0"    />
\end{lstlisting}

The mesh we make is for a simple NACA 6412 profile in a rectangular domain.

\begin{tutorialtask}
  Create a triangular mesh according to the .mcf provided in the tutorial
  folder. To make this mesh, run the command
  %
  \tutorialcommand{\$NEK/NekMesh naca.mcf naca.xml}
  %
  More details about the generation process can be seen by enabling the
  \emph{verbose} command line option \inltt{-v}:
  %
  \tutorialcommand{\$NEK/NekMesh -v naca.mcf naca.xml}
\end{tutorialtask}

This will produce a Nektar++ mesh, which we now need to visualise.

\begin{tutorialtask}
  Visualise the mesh using FieldConvert. For Tecplot output for
  visualisation in VisIt, run the command
  %
  \tutorialcommand{\$NEK/FieldConvert naca.xml naca.dat}
  %
  or, for VTK output, for visualisation in ParaView or VisIt, run the command
  %
  \tutorialcommand{\$NEK/FieldConvert naca.xml naca.vtu}
\end{tutorialtask}

As these meshes are converted to a linear mesh, since this is what the
visualisation software supports, the curved elements can look quite faceted. To
counteract this, we can add an optional \inlsh{-nX} command line option to
FieldConvert, which increases the number of subdivisions used to view the
elements through interpolation.

\begin{tutorialtask}
  Run the command
  %
  \tutorialcommand{\$NEK/FieldConvert -n 8 naca.xml naca.vtu}
  %
  and visualise the output, which should now look far smoother around the
  aerofoil's leading edge and give a better visualisation of the high-order
  elements.
\end{tutorialtask}

The .mcf has already been configured for the generation of a sufficiently fine
mesh. Among others, it includes a boundary layer mesh as well as a refinement
region in the wake of the aerofoil.
We can now try to run some simulations on this mesh. A session file
\inlsh{session\_naca.xml} for the aerofoil mesh is provided, which will execute a
low Reynolds number incompressible simulation. The resulting flow field should
look like in Fig.~\ref{f:naca}.

\begin{tutorialtask}
  Execute the solver with
  \tutorialcommand{\$NEK/IncNavierStokesSolver naca.xml session\_naca.xml}
  Process the final flow field visualisation with
  \tutorialcommand{\$NEK/FieldConvert naca.xml naca.fld naca.vtu}
  and visualise the results.
\end{tutorialtask}

\begin{figure}
  \begin{center}
    \includegraphics[width=0.5\textwidth]{img/naca.png}
    \caption{Visualisation of streamwise velocity field at $t=2.5$.}
    \label{f:naca}
  \end{center}
\end{figure}

It is possible to visualise the unsteady flow by getting the solver to write
checkpoint files every certain number of timesteps. The output frequency is
controlled by the parameter
\begin{lstlisting}[style=XMLStyle]
  <P> IO_CheckSteps = 50 </P>
\end{lstlisting}

\begin{tutorialtask}
  Rerun the solver with the checkpoint I/O turned on, by enabling the parameter
  in the parameters tag in the session file. Re-run the solver with
  %
  \tutorialcommand{\$NEK/IncNavierStokesSolver naca.xml session\_naca.xml}
  %
  Finally, process these checkpoint files for visualisation by using the command
  %
  \tutorialcommand{for i in \{0..100\}; do \$NEK/FieldConvert naca.xml
    naca\_\$\{i\}.chk naca\_\$\{i\}.vtu; done}
  %
  Note that this is a \texttt{bash} command -- if you use an alternative shell,
  such as \texttt{tcsh}, you may need to alter this syntax. Load the files into
  your visualiser as a group, and use the frame functions to view the flow as an
  unsteady animation.
\end{tutorialtask}

Following these first 2 tutorials on two-dimensional high-order mesh generation
using NekMesh, the reader should now be able to modify the mesh configuration
file according to their needs.

\begin{tutorialtask}
  Change the mesh and solver parameters to produce different meshes and flow
  results, so that the wake can be captured accurately.
\end{tutorialtask}

\end{document}
